\documentclass{scrartcl}
\usepackage[utf8]{inputenc}
\usepackage[french]{babel}
\usepackage{amsmath}
\usepackage{amsthm}
\usepackage{amssymb}
\usepackage{braket}
\usepackage{lmodern}
\usepackage[T1]{fontenc}
\setlength{\parindent}{0pt}
\usepackage{tikz}
\usepackage{hyperref}
\usepackage{float}
\usepackage[lowercase]{theoremref}
\usepackage{enumerate}
\DeclareMathAlphabet{\mathbbb}{U}{bbold}{m}{n}
\usepackage{multicol}
\usepackage[dvipsnames]{xcolor}
\renewcommand{\le}{\leqslant}
\renewcommand{\ge}{\geqslant}
\newcommand{\R}{\mathbb R}
\newcommand{\Q}{\mathbb Q}
\newcommand{\C}{\mathbb C}
\newcommand{\N}{\mathbb N}
\newcommand{\Z}{\mathbb Z}
\newcommand{\U}{\mathbb U}
\newcommand{\id}{\mathrm{id}}
\newcommand{\ind}{\mathbbb 1}
\newcommand{\clC}{\mathrm C}
\renewcommand{\O}{\mathrm O}
\renewcommand{\o}{\mathrm o}
\newcommand{\sube}{\subseteq}
\renewcommand{\H}{\mathrel{\mathcal H}}
\newcommand{\V}{\mathrel{\mathcal V}}

\newtheorem{theoreme}{Th\'eor\`eme}
\newtheorem{lemme}{Lemme}
\newtheorem{proposition}{Proposition}
\newtheorem{corollaire}{Corollaire}
\newtheorem{conjecture}{Conjecture}
\newtheorem{remarque}{Remarque}
\newtheorem{question}{Question}
\theoremstyle{definition}
\newtheorem{definition}{Définition}
\newtheorem{exemples}{Exemples}
\newtheorem{exemple}{Exemple}
\newtheorem{exercice}{Exercice}
\theoremstyle{remark}
\newtheorem{lemma}[theoreme]{Lemma}
\newtheorem{preuve}{Preuve}


\title{Pavages périodiques, apériodiques du plan}
\author{Matteo Wei et Nathan Boyer}

\begin{document}

\maketitle

\section{Introduction}

\begin{definition}[Ensemble de Wang]
    
Un ensembles de Wang est un triplet (H,V,T) où H et V sont respectivement les couleurs horizontales et verticales
et où $T \sube H^2 * V^2$ est l'ensemble des dominos

\end{definition}

\begin{definition}[Pavage]

Soit $X \sube \Z^2$ et $\tau$ un ensemble de Wang

Un pavage de X par $\tau$ est une fonction $f:X \to T$ avec:

$\forall (x,y) \in X, {f(x,y)}_e = {f(x+1,y)}_w et {f(x,y)}_n = {f(x,y+1)}_s$

Un pavage du plan par $\tau$ est un pavage de $\Z^2$

On dit que $\tau$ est périodique s'il existe un pavage du plan périodique par $\tau$

(càd que $\exists (x,y) \in Z^2 \setminus (0,0), \forall (x,y) \in \Z^2, f(x,y) = f(x+u,y+v)$)

On dit que $\tau$ est apériodique s'il existe au moins un pavage du plan par $\tau$ mais que tous ces pavages ne sont pas périodiques

\begin{conjecture}[Wang, 1961]
  Il n'existe pas d'ensemble de Wang apériodique.
\end{conjecture}
    
\end{definition}



\section{Indécidabilité du problème du domino}

\subsection{Le problème du domino}

\begin{definition}[problème du domino]
  Existe-t-il un algorithme qui, étant donné un ensemble fini de dominos de Wang, décide s'il existe un pavage du plan avec ces dominos?
\end{definition}

\begin{proposition}
  La conjecture de Wang implique que le problème du domino est décidable.
  \thlabel{prop:Wang->domino}
\end{proposition}

\begin{lemme}[compacité]
  Soit $T$ un ensemble fini de dominos de Wang. Alors $T$ pave le plan si, et seulement si, pour tout $n\in\N$, $T$ pave un carré de taille $n\times n$.
  \thlabel{lem:comp}
\end{lemme}

\begin{preuve}[\thref{lem:comp}]
  Le sens direct est trivial. Pour le sens réciproque, supposons que pour tout $n\in\N$, $T$ pave un carré de taille $n\times n$.

  On fixe un ensemble dénombrable de variables $V$ et considère le langage contenant un symbole de constante $O$, deux symboles de relations binaires $\mathcal H$ et $\mathcal V$, ainsi que, pour chaque domino $t\in T$, un symbole de relation unaire aussi noté $t$.
  
  On note $T_{\mathcal H}$ l'ensemble des couples de dominos $(t,u)\in T^2$ tels que la couleur de droite de $t$ est la couleur de gauche de $u$. De même, on note $T_{\mathcal V}$ l'ensemble des couples de dominos $(t,u)\in T^2$ tels que la couleur du haut de $t$ est la couleur du bas de $u$.

  On propose l'axiomatisation $\mathrm{Th}$ du premier ordre suivante pour les $T$-pavages:
  \begin{align}
    \tag{$\varphi_{\mathcal H}$}
    \forall x\forall y (x\H y\rightarrow\forall z((x\H z\rightarrow y=z)\land (z\H y\rightarrow x=z)))\\
    \tag{$\varphi_{\mathcal V}$}
    \forall x\forall y (x\V y\rightarrow\forall z((x\V z\rightarrow y=z) \land (z\V y\rightarrow x=z)))\\
    \tag{$\psi_{\mathrm{un}}$}
    \forall x\bigvee_{t\in T}\Bigl(t(x)\land\bigwedge_{u\in T\setminus\{t\}}\lnot u(x)\Bigr)\\
    \tag{$\psi_{\mathcal H}$}
    \forall x\forall y \Bigl( x\H y \rightarrow \bigvee_{(t,u)\in T_{\mathcal H}}(t(x)\land u(y))\Bigr)\\
    \tag{$\psi_{\mathcal V}$}
    \forall y \Bigl( x\V y \rightarrow \bigvee_{(t,u)\in T_{\mathcal V}}(t(x)\land u(y))\Bigr)
  \end{align}

  Les variables s'incarnent en des positions sur lesquelles les dominos sont placées. Les formules $\varphi_\cdot$ axiomatisent le réseau entier du plan, et les formules $\psi_\cdot$ correspondent au placement des dominos sur les cases.

  Soit $n\in\N$. Considérons
  \begin{align}
    \tag{$\theta_n$}
    x_{0,0}=O\land\bigwedge_{i=-n}^n\bigwedge_{j=-n}^{n-1} x_{i,j}\H x_{i,j+1}\land \bigwedge_{i=-n}^{n-1}\bigwedge_{j=-n}^n x_{i,j}\V x_{i+1,j}
  \end{align}
  et notons $\theta'_n=\exists(x_{i,j})_{-n\le i,j\le n}\theta_n$.

  L'hypothèse que $T$ pave des carrés de taille arbitrairement grande donne en particulier que pour tout $n\in\N$, $\mathrm{Th}\cup \{\theta'_n\}$ admet un modèle (il suffit de prendre un pavage d'un carré $C_n$ de taille $2n+1\times 2n+1$ et de prendre pour $O^{C_n}$ le centre du carré). On en déduit que $\displaystyle\mathrm{Th}\cup \set{\theta'_n|n\in\N}$ est finiment consistante, donc consistante par compacité de la logique du premier ordre. Fixons-en donc un modèle $\mathfrak M$, dont on note $M$ l'ensemble sous-jacent.

  Soit $n\in\N$. Remarquons que $\varphi_\mathcal H$ et $\varphi_\mathcal V$ impliquent que pour $n\in\N$ il n'y a qu'un seul $(a_{i,j}^n)_{-n\le i,j\le n}\in M^{(2n+1)^2}$ tel que $\mathfrak M\vdash \theta_n\left((a_{i,j})_{-n\le i,j\le n}\in M^{(2n+1)^2}\right)$, et de plus que pour $m\ge n$, et $-n\le i,j\le n$, on a $a_{i,j}^n=a_{i,j}^m$. Il s'agit en effet de l'unique position $i$ lignes au dessus et $j$ colonnes à droite de $O^\mathfrak M$; on note donc $a_{i,j}$ cet élément.

  On en déduit également que pour tout $(i,j)\in\Z^2$, $\mathfrak M\vdash a_{i,j}\H a_{i,j+1}\land a_{i,j}\V a_{i+1,j}$. On déduit alors des trois derniers axiomes de $\mathrm{Th}$ que placer à la position $(i,j)\in\Z^2$ l'unique domino $t$ telle que $\mathfrak M\vdash t(a_{i,j})$ donne un $T$-pavage du plan, ce qui conclut. 
\end{preuve}

\begin{preuve}[\thref{prop:Wang->domino}]
  Supposons que la conjecture de Wang est vrai.
  
  Considérons l'algorithme, étant donné un ensemble fini $T$ de dominos de Wang, consistant à tester, pour chaque $n\in\N$, tous les $T$-coloriages possibles de $[\![0,n]\!]^2$, jusqu'à soit trouver un $T$-pavage prolongeable en $T$-pavage périodique du plan, auquel cas on accepte, soit trouver un $n$ tel qu'aucun des $T$-coloriages de $[\![0,n]\!]^2$ n'est un $T$-pavage, auquel cas on rejette.

  Soit $T$ un ensemble fini de dominos. Il y a deux cas de figures possibles:
  \begin{itemize}
    \item Soit $T$ ne pave pas le plan, auquel cas le \thref{lem:comp} assure que l'algorithme rejette.
    \item Soit $T$ pave le plan, auquel cas la conjecture de Wang assure qu'il existe un $T$-pavage périodique, et l'algorithme accepte.
  \end{itemize}

  Ainsi, cet algorithme décide le problème du domino.
\end{preuve}

\subsection{Indécidabilité du problème du domino fixé}

\begin{definition}[problème du domino fixé]
  Existe-t-il un algorithme qui, étant donné un ensemble fini non vide $T$ de dominos de Wang et $t\in T$ décide s'il existe un pavage du quart de plan $\N^2$ avec la domino $t$ en $(0,0)$?
\end{definition}

Par une preuve similaire à celle du \thref{lem:comp}, on montre:

\begin{lemme}[compacité]
  Soit $T$ un ensemble non vide de dominos de Wang, et $t\in T$. Alors il existe un $T$-pavage de $\N^2$ avec $t$ en $(0,0)$ si, et seulement si, pour tout $n\in\N$ il existe un $T$-pavage de $[\![0,n]\!]^2$ avec $t$ en $(0,0)$.
  \thlabel{lem:comp2}
\end{lemme}

\begin{proposition}
  Le problème du domino fixé est indécidable.
  \thlabel{prop:fix}
\end{proposition}

Un calcul d'une machine de Turing peut naturellement être représenté dans le plan, et on peut facilement en déduire une manière d'encoder une machine de Turing par un ensemble de Wang. 

\begin{definition}
  Soit $\mathcal M=(Q,\Sigma,\Gamma,\Delta,q_{\mathrm{init}}, q_{\mathrm{acc}})$ une machine de Turing (non déterministe). On définit un ensemble de dominos de Wang $T_\mathcal M$, comme dans la figure \ref{fig:tiles}.
\end{definition}

\begin{figure}
  \centering
  \begin{tikzpicture}[scale = 1]
    \fill [fill = Gray] (14.5-1/8, 4-1/8) rectangle (14.5+1/8, 5);
    \fill [fill = Gray] (14.5, 4) circle (8 pt);
    \fill [fill = Black] (13.5,3) rectangle (1/4+13.5,5);
    \fill [fill = Black] (13.5,3) rectangle (15.5,3+1/4);
    \fill [fill = Black] (3,3) rectangle (5,3+1/4);
    \foreach \x in {1, 2, 4}
    \fill [fill = BrickRed] (3*\x,0) rectangle (3*\x+2,2);
    \fill [fill = Red] (6, 0) -- (7, 1) -- (8, 0) -- cycle;
    \foreach \x in {1, 2, 3, 4, 5}
    \fill [fill = BrickRed] (3*\x,-3) rectangle (3*\x+2,-1);
    \fill [fill = Red] (6, -3) -- (7, -2) -- (8, -3) -- cycle;
    \fill [fill = Red] (9, -3) -- (10, -2) -- (11, -3) -- cycle;
    \foreach \x in {1, 2, 4}
    {\fill [fill = Black] (3*\x,0) rectangle (3*\x+1/4,2);
    \fill [fill = White] (3*\x+2-1/4, 0) rectangle (3*\x+2, 2);}
    \foreach \x in {1, 2, 3, 4, 5}
    {\fill [fill = White] (3*\x,-3) rectangle (3*\x+1/4,-1);
    \fill [fill = White] (3*\x+2-1/4, -3) rectangle (3*\x+2, -1);}
    \fill [fill = RoyalBlue] (7-1/8, 0) -- (7-1/8, 1+1/8) -- (7+1/8, 1-1/8) -- (7+1/8, 0) --cycle;
    \fill [fill = Blue] (8, 1+1/8) -- (7-1/8, 1+1/8) -- (7+1/8, 1-1/8) -- (8, 1-1/8) --cycle;
    \fill [fill = Blue] (13-1/8, 1-1/8) rectangle (13+1/8, 2);
    \fill [fill = Blue] (13-1/8, 1-1/8) rectangle (14, 1+1/8);
    \fill [fill = Blue] (13, 1) circle (8 pt);
    \fill [fill = Black] (8-1/4, 3/2-1/8) rectangle (8, 3/2+1/8);
    \fill [fill = Black] (14, 1/2+1/8) -- (14-1/4, 1/2) -- (14, 1/2-1/8) -- cycle;
    \fill [fill = RoyalBlue] (7-1/8, -3) -- (7-1/8, -2+1/8) -- (7+1/8, -2-1/8) -- (7+1/8, -3) --cycle;
    \fill [fill = Blue] (8, -2+1/8) -- (7-1/8, -2+1/8) -- (7+1/8, -2-1/8) -- (8, -2-1/8) --cycle;
    \fill [fill = RoyalBlue] (10+1/8, -3) -- (10+1/8, -2+1/8) -- (10-1/8, -2-1/8) -- (10-1/8, -3) --cycle;
    \fill [fill = Blue] (9, -2+1/8) -- (10+1/8, -2+1/8) -- (10-1/8, -2-1/8) -- (9, -2-1/8) --cycle;
    \fill [fill = Black] (8-1/4, -3+3/2-1/8) rectangle (8, -3+3/2+1/8);
    \fill [fill = Black] (14, -3+1/2+1/8) -- (14-1/4, -3+1/2) -- (14, -3+1/2-1/8) -- cycle;
    \fill [fill = Black] (9+1/4, -3+1/2-1/8) rectangle (9, -3+1/2+1/8);
    \fill [fill = Black] (15, -3+3/2+1/8) -- (15+1/4, -3+3/2) -- (15, -3+3/2-1/8) -- cycle;
    \fill [fill = Blue] (13-1/8, -3+1-1/8) rectangle (13+1/8, -3+2);
    \fill [fill = Blue] (13-1/8, -3+1-1/8) rectangle (14, -3+1+1/8);
    \fill [fill = Blue] (13, -3+1) circle (8 pt);
    \fill [fill = Blue] (13-1/8, -3+1-1/8) rectangle (13+1/8, -3+2);
    \fill [fill = Blue] (13-1/8, -3+1-1/8) rectangle (14, -3+1+1/8);
    \fill [fill = Blue] (13, -3+1) circle (8 pt);
    \fill [fill = Blue] (16-1/8, -3+1-1/8) rectangle (16+1/8, -3+2);
    \fill [fill = Blue] (16+1/8, -3+1-1/8) rectangle (15, -3+1+1/8);
    \fill [fill = Blue] (16, -3+1) circle (8 pt);
    \foreach \x in {1, 2, 4}
    \draw (3*\x,0) rectangle (3*\x+2,2);
    \foreach \x in {1, 4.5}
    \draw (3*\x,3) rectangle (3*\x+2,5);
    \foreach \x in {1, 2, 3, 4, 5}
    \draw (3*\x,-3) rectangle (3*\x+2,-1);
    \node at (4, 2.5) {${\mathrm{let}}^{\mathrm S}(\_)$};
    \node at (14.5, 2.5) {$\mathrm{init}$};
    \node at (4, -0.5) {${\mathrm{let}}^\mathrm W(s)$};
    \node at (7, -0.5) {$\mathrm{trans}_\rightarrow^\mathrm W(\delta_\rightarrow)$};
    \node at (13, -0.5) {$\mathrm{head}_\leftarrow^\mathrm W(q,s)$};
    \node at (4, -3.5) {${\mathrm{let}}(s)$};
    \node at (7, -3.5) {$\mathrm{trans}_\rightarrow(\delta_\rightarrow)$};
    \node at (13, -3.5) {$\mathrm{head}_\leftarrow(q,s)$};
    \node at (10, -3.5) {$\mathrm{trans}_\leftarrow(\delta_\leftarrow)$};
    \node at (16, -3.5) {$\mathrm{head}_\rightarrow(q,s)$};
  \end{tikzpicture}
  \caption{On se donne une couleur par symbole dans l'alphabet de travail, et une autre par état de l'automate (ici, le blanc représente le caractère blanc $\_$, le gris l'état initial et le noir est une couleur supplémentaire). Les dominos sont formés de deux couches superposées: celle du fond est constituée d'aplats de couleurs de symbole, celle du dessus porte des rubans portant des couleurs d'états.\\$T_\mathcal M$ comporte un domino $\mathrm{let}(s)$ et un domino $\mathrm{let}^\mathrm W(s)$ par symbole $s\in\Gamma$, un domino $\mathrm{head}_\leftarrow(q,s)$, un domino $\mathrm{head}_\rightarrow(q,s)$ et un domino $\mathrm{head}_\leftarrow^\mathrm W(q,s)$ par couple $(q,s)\in Q\times\Gamma$, un domino $\mathrm{trans}_\rightarrow(\delta_\rightarrow)$ et un domino $\mathrm{trans}_\rightarrow^\mathrm W(\delta_\rightarrow)$ par transition droite $\delta_\rightarrow$, et un domino $\mathrm{trans}_\leftarrow(\delta_\leftarrow)$ par transition gauche $\delta_\leftarrow$.}
  \label{fig:tiles}
\end{figure}

\begin{preuve}[\thref{prop:fix}]
  Soit $\mathcal M$ une machine de Turing.

  Supposons que $T_\mathcal M$ pave $\N^2$, avec $\mathrm{init}$ en position $(0,0)$. Remarquons que la première ligne est nécessairement de la forme $\mathrm{init},\mathrm{let}^\mathrm S(\_), \mathrm{let}^\mathrm S(\_),\ldots$. Elle correspond à la configuration initiale de $\mathcal M$ sur la bande blanche.

  On prouve maintenant par récurrence sur $i\in\N$ que lire les couleurs de fond des côtés hauts des domino de la $i$-ème ligne donne le résultat après $i$ étapes de calcul de la machine de Turing sur le mot vide, et qu'il y a un unique domino de la forme $\mathrm{head}_\cdot^\cdot(\cdot,\cdot)$ (ou $\mathrm{init}$) sur la $i$-ème ligne, qu'il est à la position de la tête de lecture, et que sa couleur d'état est celle de la machine; on vient de montrer l'initialisation.

  Soit maintenant $i\in\N$ tel que ce soit vrai. Notons $q$ l'état correspondant, $s$ le symbole sur lequel la tête est, et $j$ sa colonne. Le seul domino pouvant être placé au dessus est un domino de transition, correspondant à une transition de l'état $q$ pour la lettre $s$, donc $\mathcal M$ admet une telle transition $\delta=(q,s,\tilde q,\tilde s,D)$ (et si on était déjà sur la bordure ouest, $D$ ne peut être $\leftarrow$). De plus, la couleur du haut du domino en position $(j,i+1)$ est $\tilde q$. Par ailleurs, tout autre domino de la ligne $i+1$ doit avoir la couleur de fond de celui juste en dessous de lui. Ainsi, la $i+1$-ème ligne a les couleurs du résultat de $i+1$-étapes de calcul sur le mot vide. Par ailleurs, le domino à $D$ de $(j,i+1)$ doit être de la forme $\mathrm{head}_D^\cdot(\tilde q,\cdot)$. De plus, tout domino de la forme $\mathrm{head}_{D'}^\cdot(\cdot,\cdot)$ doit nécessairement être à $-D'$ d'un domino de transition de direction $D'$, qui doit lui-même être au dessus d'un domino de la forme $\mathrm{head}_{\cdot}^\cdot(\cdot,\cdot)$. On en déduit qu'il n'y en a qu'un seul, le domino à $D$ de la position $(j, i+1)$. Cela conclut l'hérédité car il porte le bon état et est à la bonne position.

  On en déduit en particulier que $\mathcal M$ ne termine pas sur le mot vide.

  \

  Supposons maintenant que $\mathcal M$ ne termine pas sur le mot vide. En adaptant le raisonnement qu'on vient d'effectuer, on construit un pavage de $\N^2$ avec $T_\mathcal M$ tel que $\mathrm{init}$ est en $(0,0)$: on écrit une bande vide sur la première ligne, avec la tête $\mathrm{init}$ en $(0,0)$, puis ensuite on complète ligne par ligne en copiant la ligne sur la ligne du dessus, sauf pour le domino au dessus du domino de tête, où on place le domino de la transition qu'on fait, et le domino vers lequel cette transition pointe, où l'on place le domino de tête de la bonne direction, et ayant comme état le nouvel état de la machine. Comme la machine ne termine pas sur le mot vide, on construit ainsi un pavage de $\N^2$.
  
  \ 

  Ainsi, $T_\mathcal M$ pave $\N^2$ avec $\mathrm{init}$ en $(0,0)$ si, et seulement si, $\mathcal M$ ne termine pas sur le mot vide. On en déduit, comme le problème de l'arrêt est indécidable, que c'est aussi le cas du problème du problème du domino fixé.
\end{preuve}

\subsection{Vers le problème du domino}

\begin{proposition}
  Le problème du domino est indécidable.
\end{proposition}

\begin{preuve}
  On essaie de se ramener au problème du domino fixé. Pour cela, on essaie de trouver un ensemble de Wang dont tous les pavages délimitent des carrés vides de taille arbitrairement grande.
  
  Une fois qu'on a un tel ensemble $T_s$, étant donné un ensemble de Wang $T$, et un domino $t\in T$, on construit un nouvel ensemble de Wang $T'$ à partir de $T_s$ et $T$: on prend les dominos de $T_s$ et on rajoute sur une couche les dominos de $T$, de telle sorte à ce que les carrés soient pavés avec les dominos de $T$, avec $t$ au sud-ouest de chaque grand carré. Alors on peut paver le plan avec $T'$ si, et seulement si, on peut paver des carrés de taille arbitrairement grande avec $t$ à leur coin sud-ouest. Par le \thref{lem:comp2}, on en déduit que $T'$ pavee le plan si, et seulement si, $T$ pave $\N^2$ avec $t$ en $(0,0)$.

  Comme le problème du domino fixé est indécidable, on en déduit que le problème du domino l'est aussi.

  \

  La construction d'un ensemble de Wang tel que $T_s$ peut être faite à l'aide de \emph{substitutions}. On construit un ensemble de dominos $T_s$ tel qu'il existe une substitution, \emph{i.e.} une fonction $\varphi:T_s\to T_s^{\{0,1\}^2}$ tel que (l'extension naturelle aux $T_s$-coloriages du plan de) $\varphi$ préserve le fait d'être un pavage, et tel que tout $T_s$-pavage de $\Z^2$ est (à translation près) la $\varphi$-image d'un autre $T_s$-pavage.
  
  Si en plus $T_s$ fait vivre des carrés, et que $\varphi$ les fait doubler de taille, on en déduit facilement que $T_s$ vérifie les propriétés qu'on veut.

  \

  La construction de tels ensembles de Wang est décrite dans (placeholder et placeholder). Comme les ensembles construits ont beaucoup de dominos ($104$ dans les deux cas), les constructions ne seront pas précisées ici.
\end{preuve}

\subsection{Complétude pour plusieurs classes de complexité de problèmes de pavages}

L'idée générale est d'utiliser le codage des machines de Turing en pavage pour ramener une instance d'un problème donné à une instance d'un problème de pavage.  

\begin{definition}[problèmes $\mathrm{TILING}$]
  Soit $h:\N\to\N$ et $w:\N\to\N$ tel que $w\ge\id_\N$. Le problème de décision $\mathrm{TILING}(h,w)$, consiste, étant donné un ensemble de Wang $T$, quatre couleurs $N$, $S$, $E$ et $W$, et une suite de couleurs de longueur $n$, consiste à décider s'il existe un $T$-pavage d'un carré de taille $w(n)\times h(n)$ avec les contraintes supplémentaires que les bords nord, est et ouest soient des couleurs correspondantes et que le côté sud commence par la suite de couleurs donnée, puis continue avec la couleur $S$.

  On note $\mathrm{TILING}(*,w)$, le même problème, où l'on enlève les contraintes de hauteur.
\end{definition}

\begin{definition}
  Soit $\mathcal M=(Q,\Sigma,\Gamma, \Delta, q_\mathrm{init}, q_\mathrm{acc})$ une machine de Turing. On construit un ensemble de Wang $T^{\mathrm{born}}_\mathcal M$ comme dans la figure \ref{fig:tiles2}.
\end{definition}

\begin{figure}
  \centering
  \begin{tikzpicture}[scale = 0.8]
    \foreach \x in {1, 2, 3, 4, 5, 6}
    \fill [fill = BrickRed] (3*\x,-6) rectangle (3*\x+2,-4);
    \fill [fill = Red] (6, -6) -- (7, -5) -- (8, -6) -- cycle;
    \fill [fill = Red] (9, -6) -- (10, -5) -- (11, -6) -- cycle;
    \foreach \x in {1, 6}
    \fill [fill = BrickRed] (3*\x,-3) rectangle (3*\x+2,-1);
    \foreach \x in {1, 2, 3, 4, 5, 6}
    {\fill [fill = White] (3*\x,-6) rectangle (3*\x+1/4,-4);
    \fill [fill = White] (3*\x+2-1/4, -6) rectangle (3*\x+2, -4);}
    \fill [fill = RoyalBlue] (7-1/8, -6) -- (7-1/8, -5+1/8) -- (7+1/8, -5-1/8) -- (7+1/8, -6) --cycle;
    \fill [fill = Blue] (8, -5+1/8) -- (7-1/8, -5+1/8) -- (7+1/8, -5-1/8) -- (8, -5-1/8) --cycle;
    \fill [fill = RoyalBlue] (10+1/8, -6) -- (10+1/8, -5+1/8) -- (10-1/8, -5-1/8) -- (10-1/8, -6) --cycle;
    \fill [fill = Blue] (9, -5+1/8) -- (10+1/8, -5+1/8) -- (10-1/8, -5-1/8) -- (9, -5-1/8) --cycle;
    \fill [fill = Black] (9+1/4, -6+1/2-1/8) rectangle (9, -6+1/2+1/8);
    \fill [fill = Black] (14, -6+1/2+1/8) -- (14-1/4, -6+1/2) -- (14, -6+1/2-1/8) -- cycle;
    \fill [fill = Black] (9+1/4, -6+1/2-1/8) rectangle (9, -6+1/2+1/8);
    \fill [fill = Black] (15, -6+3/2+1/8) -- (15+1/4, -6+3/2) -- (15, -6+3/2-1/8) -- cycle;
    \fill [fill = Blue] (13-1/8, -6+1-1/8) rectangle (13+1/8, -6+2);
    \fill [fill = Blue] (13-1/8, -6+1-1/8) rectangle (14, -6+1+1/8);
    \fill [fill = Blue] (13, -6+1) circle (8 pt);
    \fill [fill = Blue] (13-1/8, -6+1-1/8) rectangle (13+1/8, -6+2);
    \fill [fill = Blue] (13-1/8, -6+1-1/8) rectangle (14, -6+1+1/8);
    \fill [fill = Blue] (13, -6+1) circle (8 pt);
    \fill [fill = Blue] (16-1/8, -6+1-1/8) rectangle (16+1/8, -6+2);
    \fill [fill = Blue] (16+1/8, -6+1-1/8) rectangle (15, -6+1+1/8);
    \fill [fill = Blue] (16, -6+1) circle (8 pt);
    \foreach \x in {1, 6}
    {\fill [fill = White] (3*\x,-3) rectangle (3*\x+1/4,-1);
    \fill [fill = White] (3*\x+2-1/4, -3) rectangle (3*\x+2, -1);
    \fill [fill = Gray] (3*\x, -1) -- (3*\x+1/4, -1-1/4) -- (3*\x+2-1/4, -1-1/4) -- (3*\x+2, -1) -- cycle;
    \draw (3*\x,-3) rectangle (3*\x+2,-1);}
    \foreach \y in {-1, -2}
    {\fill [fill = Black] (19-1/8, 3*\y) rectangle (19+1/8, 3*\y+1);
    \fill [fill = Black] (19, 3*\y+1) circle (8 pt);}
    \foreach \x in {1, 2, 3, 4, 5, 6}
    \draw (3*\x,-6) rectangle (3*\x+2,-4);
    \node at (19, -6.5) {$\mathrm{acc}(s)$};
    \node at (4, -6.5) {${\mathrm{let}}(s)$};
    \node at (19, -3.5) {$\mathrm{acc}^\mathrm N(s)$};
    \node at (4, -3.5) {${\mathrm{let}}^\mathrm N(s)$};
    \node at (7, -6.5) {$\mathrm{trans}_\rightarrow(\delta_\rightarrow)$};
    \node at (13, -6.5) {$\mathrm{head}_\leftarrow(q,s)$};
    \node at (10, -6.5) {$\mathrm{trans}_\leftarrow(\delta_\leftarrow)$};
    \node at (16, -6.5) {$\mathrm{head}_\rightarrow(q,s)$};
  \end{tikzpicture}
  \caption{Les dominos de $T_{\mathcal M}^{\mathrm{born}}$. L'état noir est l'état acceptant $q_{\mathrm{acc}}$. \\ La bande grise marque le côté nord, et le blanc est le même que celui du caractère spécial $\_$.}
  \label{fig:tiles2}
\end{figure}

La possibilité de traduire des résultats des calculs de machine de Turing assure que les problèmes de pavages capturent de nombreuses classes de complexité au dessus de $P$.

\begin{proposition}
  Soit $k\in\N$. $\mathrm{TILING}(\exp_k,\exp_k)$ est $k\mathrm{NEXPTIME}$-complet (pour des réductions polynomiales).
  \thlabel{prop:Nexp}
\end{proposition}

\begin{preuve}[\thref{prop:Nexp}]
  Soit $T$ un ensemble de Wang. Étant donné une suite de couleur de longueur $n$, on décide de manière non-déterministe s'il existe un $T$-pavage suivant les conditions coulus d'un carré de taille $\exp_k(n)$ en en devinant un. Ainsi, $\mathrm{TILING}(\exp_k,\exp_k)$ est dans $k\mathrm{NEXPTIME}$.

  \

  Soit maintenant $P$ un problème de décision de $k\mathrm{NEXPTIME}$. On peut trouver une machine de Turing (non déterministe) $\mathcal M$ décidant $P$, en au plus $exp_k(f(n))$ pour une entrée de taille $n$, où $f$ est un certain polynôme.

  Considérons $T^\mathrm{born}_{\mathcal M}$, où les couleurs $S$, $E$ et $W$ sont blanc et $N$ est gris. Remarquons que, $\mathcal M$ accepte un mot $w$ si, et seulement si, $T^\mathrm{born}_\mathcal M$ pave un carré de taille assez grande pour faire tenir tous les calculs, en imposant en plus que la couleur à l'extrémité ouest du bord sud soit $w_0$, avec un marqueur de tête à l'état $q_\mathrm{init}$ dessus, suivie dans l'ordre des couleurs $w_i$, puis de blancs.

  C'est en particulier le cas pour le carré de taille $\exp_k(f(|w|))$, pour la suite de couleur imposée de longueur $f(|w|)$ obtenue en concaténant celle qu'on vient d'évoquer avec $f(|w|)-|w|$ blancs.

  Ainsi, on sait réduire toute instance de $P$ en une instance de $\mathrm{TILING}(\exp_k,\exp_k)$ de taille $f(|w|)$, ce qui prouve que le problème est $k\mathrm{NEXPTIME}$-compliqué, et conclut.
\end{preuve}

\begin{proposition}
  Soit $k\in\N$. $\mathrm{TILING}(*,\exp_k)$ est $k\mathrm{EXPSPACE}$-complet (pour des réductions polynomiales).
  \thlabel{prop:expsp}
\end{proposition}

\begin{preuve}[\thref{prop:expsp}]
  Soit $T$ un ensemble de Wang. Étant donné une suite de couleur de longueur $n$, on décide de manière non-déterministe s'il existe un $T$-pavage d'un rectangle de largeur $exp_k(n)$ vérifiant les conditions imposées en en devinant un ligne par ligne; on remarque qu'il n'est nécessaire de garder en mémoire que la dernière ligne construite, et de remplacer les dominos un par un quand on construit la suivante (les conditions sur le pavage sont locales). Ainsi, $\mathrm{TILING}(*,\exp_k)$ est dans $k\mathrm{EXPSPACE}$.

  \

  La réduction pour l'autre sens est la même qu'avant.
\end{preuve}

\begin{corollaire}[\thref{prop:Nexp} et \thref{prop:expsp}]
  En particulier, $\mathrm{TILING}(\id_\N,\id_\N)$ est $\mathrm{NP}$-complet, et $\mathrm{TILING}(*,\id_\N)$ est $\mathrm{PSPACE}$-complet. 
  \thlabel{cor:P}
\end{corollaire}

\begin{preuve}[\thref{cor:P}]
  Il s'agit du cas $k=0$ de respectivement la \thref{prop:Nexp} et la \thref{prop:expsp}.
\end{preuve}

\section{Existence d'un ensemble de Wang apériodique de 11 tuiles}

On cherche à montrer dans cette partie qu'il n'existe pas d'ensemble de Wang apériodique à 11 dominos ou moins
mais qu'il en existe à partir de 11 dominos

\subsection{Définitions préalables}

\begin{definition}[Transducteur]
    
Un transducteur $\tau$ est un automate qui lit une bande d'entrée bifinie et écrit sur une bande de sortie bifinie

\end{definition}

On peut voir un pavage comme un transducteur.

En effet, $\forall t = (w,e,s,n) \in T$, on dit qu'il y a une transition de l'état w vers l'état e qui lit n et écrit s

\begin{definition}[$w \tau w'$]
    
On dit que $w \tau w'$ si $w'$ est une bande de sortie pour la bande d'entrée $w$ et le transducteur $\tau$


\end{definition}

\begin{definition}[Run]
    
Une run d'un transducteur $\tau$ sur I un intervalle de $\Z$ est une suite de mots bifinis ${(w_i)}_{i \in I}$
telle que $\forall i \in I, w_i \tau w_{i+1}$

\end{definition}

On définit également la composition de 2 transducteurs de manière naturelle

On peut alors reformuler le problème initial de la façon suivante:

\begin{proposition}
    
Un ensemble de Wang admet un pavage périodique si et seulement si $\exists w \text{ mot bifini }, k \in \N, \text{ avec } w \tau^k w'$

\end{proposition}

\begin{definition}[Equivalence de 2 ensembles de Wang]
    
On ne s'intéresse plus aux ensembles de Wang que sous le point de vue des transducteurs

Ainsi, 2 ensembles de Wang sont équivalents ssi leurs transducteurs définissent la même relation

càd ssi $\forall w,w', \; w \tau w' \iff w \tau' w'$


\end{definition}

\begin{definition}[Simulation d'un transducteur]

Soit $\tau = (H,V,T)$ un transducteur

Une simulation R est une relation sur H telle que 
    
$uRu' \iff \forall v \in H, a,b \in V \text{ avec } (u,v,a,b) \in T, \exists v' \in V, (u',v',a,b) \in T$

Une bisimulation est simulation R telle que $R^{-1}$ soit une relation

La relation de bisimilarité (qui est la plus grande bisimulation) est une relation d'équivalence.

\end{definition}

\subsection{Générer tous les ensembles de Wang de cardinal moins de 10}

Soit $\tau$ un ensemble de Wang. On dit que $\tau$ est minimal apériodique si $\tau$ est apériodique
et qu'aucun sous-ensemble strict de $\tau$ n'est apériodique.

\;

Dans toute cette partie, on va s'intéresser au graphe orienté G = (V,E) lié au transducteur de $\tau$.

On a alors les résultats suivants:

\begin{lemme}

\begin{itemize}
    \item Soit $u,v \in V \text{ avec } uv \in E \text{ mais pas de chemin de v vers u}$, alors $\tau$ n'est pas apériodique minimal
    \item Si G a une composante fortement connexe qui est un cycle, alors $\tau$ n'est pas apériodique minimal
    \item Si G n'a qu'un sommet, alors $\tau$ n'est pas apériodique
    \item Si $|E|-|V| \le 2$, alors $\tau$ n'est pas apériodique minimal
\end{itemize}

\end{lemme}

\begin{preuve}
    
Pour le premier point, (uv) correspond à un domino t qui ne peut apparaître qu'une fois par colonne,
donc, si $\tau$ pave le plan, alors $\tau \setminus \{t \}$ aussi, d'où le résultat.

\;

Pour le deuxième point, supposons qu'il existe une telle composante. Cela veut donc dire qu'il existe $S \sube T$
tel qu'à chaque fois qu'un des dominos apparaît, la ligne entière est périodique

Si T est apériodique, on ne peut pas avoir un pavage où les dominos de S apparaissent dans deux rangées différentes, car on pourrait en déduire un pavage périodique.

En conséquence, les tuiles de S apparaissent dans au plus une ligne, et en utilisant le même argument que précédemment, nous déduisons que $T \setminus S$ pave le plan.

\;

Pour le troisième point, si G n'a qu'un sommet mais que $\tau$ pave le plan, alors on peut construire un pavage
du plan tel que chaque colonne est la même.

\;

La preuve du dernier point est très longue (plus de 15 pages) et ne sera donc pas détaillée ici.

\end{preuve}

On peut également supposer que G ne contient pas de sommet isolé.

On génère alors tous les graphes avec au plus 10 arêtes et vérifiant ces différnetes propriétés par ordinateur
et on trouve qu'il en existe 77809.

\subsection{Tester l'apériodicité}

On veut maintenant tester l'apériodicité de ces différents transducteurs.

Cependant, on sait qu'un transducteur $\tau$ n'est pas apériodique si:

\begin{itemize}
    \item $\exists k \in \N$ tel que $\nexists w,w'$ avec $w \tau^k w'$ ($\tau$ ne pave pas le plan)
    \item $\exists k \in \N$ tel que $\exists w$ avec $w \tau^k w$ ($\tau$ est périodique)
\end{itemize}

L'algorithme est alors évident: Pour chaque $k \in \N$, calculer $\tau^k$ et tester si il y a un des 2 cas précédents,
jusqu'à ce que l'ordinateur ne dispose plus d'assez de mémoire (en effet, comme on l'a vu plus haut, ce problème est indécidable).
Dans ce cas, on doit alors étudier le cas à la main.

On rajoute cependant une petite optimisation, qui est d'enlever les puits et sources des graphes à chaque étape pour les simplifier
(les puits et les sources ne pouvant pas servir à grand chose quand on veut créer une suite bifinie).

A l'exception d'un cas, tous les cas furent résolus par l'ordinateur, et ainsi le résultat recherché fut prouvé.

\subsection{Un contre-exemple pour 11 tuiles}

En appliquant le même algorithme pour 11 tuiles, on peut regarder les cas indécidés par l'ordinateur et les étudier.

Un cas en particulier a alors été exhibé et démontré comme apériodique, prouvant qu'il existe des ensembles de Wang
apériodiques à 11 dominos


\end{document}