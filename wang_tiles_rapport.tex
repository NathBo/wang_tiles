\documentclass{scrartcl}
\usepackage[utf8]{inputenc}
\usepackage[french]{babel}
\usepackage{amsmath}
\usepackage{amssymb}
\usepackage{lmodern}
\usepackage[T1]{fontenc}
\setlength{\parindent}{0pt}
\usepackage{tikz}
\usepackage{hyperref}
\usepackage{float}
\usepackage[lowercase]{theoremref}
\usepackage{enumerate}
\DeclareMathAlphabet{\mathbbb}{U}{bbold}{m}{n}
\usepackage{multicol}
\renewcommand{\le}{\leqslant}
\renewcommand{\ge}{\geqslant}
\newcommand{\R}{\mathbb R}
\newcommand{\Q}{\mathbb Q}
\newcommand{\C}{\mathbb C}
\newcommand{\N}{\mathbb N}
\newcommand{\Z}{\mathbb Z}
\newcommand{\U}{\mathbb U}
\newcommand{\id}{\mathrm{id}}
\newcommand{\ind}{\mathbbb 1}
\newcommand{\clC}{\mathrm C}
\renewcommand{\O}{\mathrm O}
\renewcommand{\o}{\mathrm o}
\newcommand{\sube}{\subseteq}

\newtheorem{theoreme}{Th\'eor\`eme}
\newtheorem{lemme}{Lemme}
\newtheorem{proposition}{Proposition}
\newtheorem{corollaire}{Corollaire}
\newtheorem*{remarque}{Remarque}
\newtheorem*{question}{Question}
\theoremstyle{definition}
\newtheorem*{definition}{Définition}
\newtheorem*{exemples}{Exemples}
\newtheorem*{exemple}{Exemple}
\newtheorem{exercice}{Exercice}
\theoremstyle{remark}
\newtheorem*{preuve}{Preuve}

\title{Pavages périodiques apériodiques du plan}
\author{Matteo Wei et Nathan Boyer}

\begin{document}

\maketitle

\section{Introduction}

\section{Indécidabilité du problème du domino}

\subsection{Le problème du domino}

\begin{definition}[Problème du domino]
  Existe-t-il un algorithme qui, étant donné un ensemble fini de tuiles de Wang, décide s'il existe un pavage du plan avec ces tuiles?
\end{definition}

\begin{proposition}
  La conjecture de Wang implique que le problème du domino est décidable.
  \thlabel{prop:Wang->domino}
\end{proposition}

\begin{lemme}[compacité]
  Soit $T$ un ensemble fini de tuiles de Wang. Alors $T$ pave le plan si, et seulement si, pour tout $n\in\N$ $T$ pave un carré de taille $n\times n$.
  \thlabel{lem:comp}
\end{lemme}

\begin{preuve}[\thref{lem:comp}]
  On fixe un ensemble dénombrable de variables $V$ et considère le langage contenant un symbole de constante $O$, deux symboles de relations binaires $\mathcal H$ et $\mathcal V$, ainsi que, pour chaque tuile $t\in T$, un symbole de relation unaire aussi noté $t$.
  
  On note $T_{\mathcal H}$ l'ensemble des couples de tuiles $(t,u)\in T^2$ tels que la couleur de droite de $t$ est la couleur de gauche de $u$. De même, on note $T_{\mathcal V}$ l'ensemble des couples de tuiles $(t,u)\in T^2$ tels que la couleur du haut de $t$ est la couleur du bas de $u$.

  On propose l'axiomatisation du premier ordre suivante pour les $T$-pavages:
\end{preuve}

\section{Pavage apériodique pour 11 tuiles et 4 couleurs}


\end{document}