\documentclass{scrartcl}
\usepackage[utf8]{inputenc}
\usepackage[french]{babel}
\usepackage{amsmath}
\usepackage{amsthm}
\usepackage{amssymb}
\usepackage{amsthm}
\usepackage{braket}
\usepackage{lmodern}
\usepackage[T1]{fontenc}
\setlength{\parindent}{0pt}
\usepackage{tikz}
\usepackage{hyperref}
\usepackage{float}
\usepackage[lowercase]{theoremref}
\usepackage{enumerate}
\DeclareMathAlphabet{\mathbbb}{U}{bbold}{m}{n}
\usepackage{multicol}
\renewcommand{\le}{\leqslant}
\renewcommand{\ge}{\geqslant}
\newcommand{\R}{\mathbb R}
\newcommand{\Q}{\mathbb Q}
\newcommand{\C}{\mathbb C}
\newcommand{\N}{\mathbb N}
\newcommand{\Z}{\mathbb Z}
\newcommand{\U}{\mathbb U}
\newcommand{\id}{\mathrm{id}}
\newcommand{\ind}{\mathbbb 1}
\newcommand{\clC}{\mathrm C}
\renewcommand{\O}{\mathrm O}
\renewcommand{\o}{\mathrm o}
\newcommand{\sube}{\subseteq}
\renewcommand{\H}{\mathrel{\mathcal H}}
\newcommand{\V}{\mathrel{\mathcal V}}

\newtheorem{theoreme}{Th\'eor\`eme}
\newtheorem{lemme}{Lemme}
\newtheorem{proposition}{Proposition}
\newtheorem{corollaire}{Corollaire}
\newtheorem{remarque}{Remarque}
\newtheorem{question}{Question}
\theoremstyle{definition}
\newtheorem{definition}{Définition}
\newtheorem{exemples}{Exemples}
\newtheorem{exemple}{Exemple}
\newtheorem{exercice}{Exercice}
\theoremstyle{remark}
\newtheorem{preuve}{Preuve}


\title{Pavages périodiques apériodiques du plan}
\author{Matteo Wei et Nathan Boyer}

\begin{document}

\maketitle

\section{Introduction}

\begin{definition}
    
Un ensembles de Wang est un triplet (H,V,T) où H et V sont respectivement les couleurs horizontales et verticales

et où $T \sube H^2 * V^2$ est l'ensemble des dominos

\end{definition}

\section{Indécidabilité du problème du domino}

\subsection{Le problème du domino}

\begin{definition}[Problème du domino]
  Existe-t-il un algorithme qui, étant donné un ensemble fini de tuiles de Wang, décide s'il existe un pavage du plan avec ces tuiles?
\end{definition}

\begin{proposition}
  La conjecture de Wang implique que le problème du domino est décidable.
  \thlabel{prop:Wang->domino}
\end{proposition}

\begin{lemme}[compacité]
  Soit $T$ un ensemble fini de tuiles de Wang. Alors $T$ pave le plan si, et seulement si, pour tout $n\in\N$, $T$ pave un carré de taille $n\times n$.
  \thlabel{lem:comp}
\end{lemme}

\begin{preuve}[\thref{lem:comp}]
  Le sens direct est trivial. Pour le sens réciproque, supposons que pour tout $n\in\N$, $T$ pave un carré de taille $n\times n$.

  On fixe un ensemble dénombrable de variables $V$ et considère le langage contenant un symbole de constante $O$, deux symboles de relations binaires $\mathcal H$ et $\mathcal V$, ainsi que, pour chaque tuile $t\in T$, un symbole de relation unaire aussi noté $t$.
  
  On note $T_{\mathcal H}$ l'ensemble des couples de tuiles $(t,u)\in T^2$ tels que la couleur de droite de $t$ est la couleur de gauche de $u$. De même, on note $T_{\mathcal V}$ l'ensemble des couples de tuiles $(t,u)\in T^2$ tels que la couleur du haut de $t$ est la couleur du bas de $u$.

  On propose l'axiomatisation $\mathrm{Th}$ du premier ordre suivante pour les $T$-pavages:
  \begin{align}
    \tag{$\varphi_{\mathcal H}$}
    \forall x\forall y (x\H y\rightarrow\forall z((x\H z\rightarrow y=z)\land (z\H y\rightarrow x=z)))\\
    \tag{$\varphi_{\mathcal V}$}
    \forall x\forall y (x\V y\rightarrow\forall z((x\V z\rightarrow y=z) \land (z\V y\rightarrow x=z)))\\
    \tag{$\psi_{\mathrm{un}}$}
    \forall x\bigvee_{t\in T}\Bigl(t(x)\land\bigwedge_{u\in T\setminus\{t\}}\lnot u(x)\Bigr)\\
    \tag{$\psi_{\mathcal H}$}
    \forall x\forall y \Bigl( x\H y \rightarrow \bigvee_{(t,u)\in T_{\mathcal H}}(t(x)\land u(y))\Bigr)\\
    \tag{$\psi_{\mathcal V}$}
    \forall y \Bigl( x\V y \rightarrow \bigvee_{(t,u)\in T_{\mathcal V}}(t(x)\land u(y))\Bigr)
  \end{align}

  Les variables s'incarnent en des positions sur lesquelles les tuiles sont placées. Les formules $\varphi_\cdot$ axiomatisent le réseau entier du plan, et les formules $\psi_\cdot$ correspondent au placement des tuiles sur les cases.

  Soit $n\in\N$. Considérons
  \begin{align}
    \tag{$\theta_n$}
    x_{0,0}=O\land\bigwedge_{i=-n}^n\bigwedge_{j=-n}^{n-1} x_{i,j}\H x_{i,j+1}\land \bigwedge_{i=-n}^{n-1}\bigwedge_{j=-n}^n x_{i,j}\V x_{i+1,j}
  \end{align}
  et notons $\theta'_n=\exists(x_{i,j})_{-n\le i,j\le n}\theta_n$.

  L'hypothèse que $T$ pave des carrés de taille arbitrairement grande donne en particulier que pour tout $n\in\N$, $\mathrm{Th}\cup \{\theta'_n\}$ admet un modèle (il suffit de prendre un pavage d'un carré $C_n$ de taille $2n+1\times 2n+1$ et de prendre pour $O^{C_n}$ le centre du carré). On en déduit que $\displaystyle\mathrm{Th}\cup \set{\theta'_n|n\in\N}$ est finiment consistante, donc consistante par compacité de la logique du premier ordre. Fixons-en donc un modèle $\mathfrak M$, dont on note $M$ l'ensemble sous-jacent.

  Soit $n\in\N$. Remarquons que $\varphi_\mathcal H$ et $\varphi_\mathcal V$ impliquent que pour $n\in\N$ il n'y a qu'un seul $(a_{i,j}^n)_{-n\le i,j\le n}\in M^{(2n+1)^2}$ tel que $\mathfrak M\vdash \theta_n\left((a_{i,j})_{-n\le i,j\le n}\in M^{(2n+1)^2}\right)$, et de plus que pour $m\ge n$, et $-n\le i,j\le n$, on a $a_{i,j}^n=a_{i,j}^m$. Il s'agit en effet de l'unique position $i$ lignes au dessus et $j$ colonnes à droite de $O^\mathfrak M$; on note donc $a_{i,j}$ cet élément.

  On en déduit également que pour tout $(i,j)\in\Z^2$, $\mathfrak M\vdash a_{i,j}\H a_{i,j+1}\land a_{i,j}\V a_{i+1,j}$. On déduit alors des trois derniers axiomes de $\mathrm{Th}$ que placer à la position $(i,j)\in\Z^2$ l'unique tuile $t$ telle que $\mathfrak M\vdash t(a_{i,j})$ donne un $T$-pavage du plan, ce qui conclut. 
\end{preuve}

\begin{preuve}[\thref{prop:Wang->domino}]
  Supposons que la conjecture de Wang est vrai.
  
  Considérons l'algorithme, étant donné un ensemble fini $T$ de tuiles de Wang, consistant à tester, pour chaque $n\in\N$, tous les $T$-coloriages possibles de $[\![0,n]\!]^2$, jusqu'à soit trouver un $T$-pavage prolongeable en $T$-pavage périodique du plan, auquel cas on accepte, soit trouver un $n$ tel qu'aucun des $T$-coloriages de $[\![0,n]\!]^2$ n'est un $T$-pavage, auquel cas on rejette.

  Soit $T$ un ensemble fini de tuiles. Il y a deux cas de figures possibles:
  \begin{itemize}
    \item Soit $T$ ne pave pas le plan, auquel cas le \thref{lem:comp} assure que l'algorithme rejette.
    \item Soit $T$ pave le plan, auquel cas la conjecture de Wang assure qu'il existe un $T$-pavage périodique, et l'algorithme accepte.
  \end{itemize}

  Ainsi, cet algorithme décide le problème du domino.
\end{preuve}

\subsection{}

\section{Pavage apériodique pour 11 tuiles et 4 couleurs}

\begin{definition}[Transducteur]
    
Un transducteur $\tau$ est un automate qui lit une bande d'entrée bifinie et écrit sur une bande de sortie bifinie

\end{definition}

On peut voir un pavage comme un transducteur.

En effet, $\forall t = (w,e,s,n) \in T$, on dit qu'il y a une transition de l'état w vers l'état e qui lit n et écrit s

\begin{definition}[$w \tau w'$]
    
On dit que $w \tau w'$ si $w'$ est une bande de sortie pour la bande d'entrée $w$ et le transducteur $\tau$


\end{definition}

\begin{definition}[Run]
    
Une run d'un transducteur $\tau$ sur I un intervalle de $\Z$ est une suite de mots bifinis ${(w_i)}_{i \in I}$
telle que $\forall i \in I, w_i \tau w_{i+1}$

\end{definition}

On définit également la composition de 2 transducteurs de manière naturelle

On peut alors reformuler le problème initial de la façon suivante:

\begin{proposition}
    
Un ensemble de Wang admet un pavage périodique si et seulement si $\exists w \text{ mot bifini }, k \in \N, \text{ avec } w \tau^k w'$

\end{proposition}

\begin{definition}[Equivalence de 2 ensembles de Wang]
    
On ne s'intéresse plus aux ensembles de Wang que sous le point de vue des transducteurs

Ainsi, 2 ensembles de Wang sont équivalents ssi leurs transducteurs définissent la même relation

càd ssi $\forall w,w', \; w \tau w' \iff w \tau' w'$


\end{definition}

\begin{definition}[Simulation d'un transducteur]

Soit $\tau = (H,V,T)$ un transducteur

Une simulation R est une relation sur H telle que 
    
$uRu' \iff \forall v \in H, a,b \in V \text{ avec } (u,v,a,b) \in T, \exists v' \in V, (u',v',a,b) \in T$

Une bisimulation est simulation R telle que $R^{-1}$ soit une relation

La relation de bisimilarité (qui est la plus grande bisimulation) est une relation d'équivalence.

\end{definition}



\end{document}