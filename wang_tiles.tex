\documentclass{scrartcl}
\usepackage[utf8]{inputenc}
\usepackage[french]{babel}
\usepackage{amsmath}
\usepackage{amssymb}
\usepackage{lmodern}
\usepackage[T1]{fontenc}
\setlength{\parindent}{0pt}
\usepackage{tikz}
\usepackage{hyperref}
\usepackage{float}
\usepackage[lowercase]{theoremref}
\usepackage{enumerate}
\DeclareMathAlphabet{\mathbbb}{U}{bbold}{m}{n}
\usepackage{multicol}
\renewcommand{\le}{\leqslant}
\renewcommand{\ge}{\geqslant}
\newcommand{\R}{\mathbb R}
\newcommand{\Q}{\mathbb Q}
\newcommand{\C}{\mathbb C}
\newcommand{\N}{\mathbb N}
\newcommand{\Z}{\mathbb Z}
\newcommand{\U}{\mathbb U}
\newcommand{\id}{\mathrm{id}}
\newcommand{\ind}{\mathbbb 1}
\newcommand{\clC}{\mathrm C}
\renewcommand{\O}{\mathrm O}
\renewcommand{\o}{\mathrm o}
\newcommand{\sube}{\subseteq}

\title{Pavages périodiques apériodiques du plan}
\author{Matteo Wei et Nathan Boyer}

\begin{document}
    
    \maketitle

    \section*{Introduction}

    Explication du problème des tuiles de Wang.

    Présentation de quelques résultats de base.

    \section{Indécidabilité du problème du pavage périodique}

    On explique la preuve `Two-by-two Substitution Systems and the Undecidability of the Domino Problem'\footnote[1]{\url{https://hal.science/file/index/docid/260112/filename/sutica.pdf}}

    \section{Pavage apériodique pour 11 tuiles et 4 couleurs}

    On prouve algorithmiquement qu'il n'existe pas d'ensemble de 8 tuiles pour 4 couleurs tels qu'il existe un pavage
    apériodique mais pas de pavage périodique\footnote[2]{\url{https://arxiv.org/pdf/1506.06492.pdf}}.

    On exhibe l'algorithme pour 10 tuiles.

    On exhibe qu'il faut également au moins 4 couleurs.

    On parle rapidement de l'exemple trouvé algorithmiquement pour 11 tuiles et 4 couleurs.\footnote[3]{\url{https://www.sciencedirect.com/science/article/pii/S0304397514004393}}


\end{document}

