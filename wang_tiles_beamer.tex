\documentclass{beamer}
\title{Le problème des dominos de Wang}
\author{Matteo Wei et Nathan Boyer}
\date{2024}
\graphicspath{ {./images/} }
\AtBeginSection[]

\usepackage{graphicx}

\renewcommand{\le}{\leqslant}
\renewcommand{\ge}{\geqslant}
\newcommand{\R}{\mathbb R}
\newcommand{\Q}{\mathbb Q}
\newcommand{\C}{\mathbb C}
\newcommand{\N}{\mathbb N}
\newcommand{\Z}{\mathbb Z}
\newcommand{\U}{\mathbb U}
\newcommand{\id}{\mathrm{id}}
\newcommand{\ind}{\mathbbb 1}
\newcommand{\clC}{\mathrm C}
\renewcommand{\O}{\mathrm O}
\renewcommand{\o}{\mathrm o}
\newcommand{\sube}{\subseteq}
\renewcommand{\H}{\mathrel{\mathcal H}}
\newcommand{\V}{\mathrel{\mathcal V}}


\begin{document}

\frame{\titlepage}

\section{Introduction}

\begin{frame}
    \frametitle{Table des matières}
    \tableofcontents[currentsection]
  \end{frame}
  
\begin{frame}
\frametitle{Ensemble de Wang}

\begin{alertblock}{Definition}

    Un \emph{ensemble de Wang} est un triplet $(H,V,T)$ où $H$ et $V$ sont respectivement les couleurs horizontales et verticales
    et où $T \sube H^2 * V^2$ est l'ensemble des dominos. On appelera parfois aussi abusivement ensemble de Wang l'ensemble des dominos $T$.
    
\end{alertblock}

\includegraphics{ensemble_de_wang_exemple}

\end{frame}

\begin{frame}
\frametitle{Pavage}

\begin{alertblock}{Definition}

    Soit $X \sube \Z^2$ et $\tau$ un ensemble de Wang.

    Un pavage de X par $\tau$ est une fonction $f:X \to T$ avec:
    
    \[\forall (x,y) \in X, {f(x,y)}_e = {f(x+1,y)}_w \land {f(x,y)}_n = {f(x,y+1)}_s.\]
    
    Un pavage du plan par $\tau$ est un pavage de $\Z^2$.
    
\end{alertblock}

\begin{figure}

    \includegraphics[scale = 0.3]{pavage_exemple}
    \centering
    
\end{figure}


\end{frame}

\begin{frame}
\frametitle{Pavage périodique et apériodique}

\begin{alertblock}{Definition}

On dit que $\tau$ est périodique s'il existe un pavage du plan périodique par $\tau$
(càd tel que $\exists (u,v) \in {\Z^*}^2, \forall (x,y) \in \Z^2, f(x,y) = f(x+u,y)=f(x,y+v)$).

On dit que $\tau$ est apériodique s'il existe au moins un pavage du plan par $\tau$ mais que tous ses pavages ne sont pas périodiques.
    
\end{alertblock}

\begin{figure}

    \includegraphics[scale = 0.5]{pavage_periodique}
    \centering
    
\end{figure}

\end{frame}

\section{Indécidabilité du problème du domino}

\begin{frame}
    \frametitle{Table des matières}
    \tableofcontents[currentsection]
  \end{frame}

\section{Nombre minimal de dominos pour un ensemble de Wang apériodique}

\begin{frame}
    \frametitle{Table des matières}
    \tableofcontents[currentsection]
  \end{frame}

\subsection{Définitions préalables}

\begin{frame}
\frametitle{Transducteur}

\begin{alertblock}{Definition}

Un transducteur $\tau$ est un automate qui lit une bande d'entrée bifinie et écrit sur une bande de sortie bifinie.
    
\end{alertblock}

\begin{figure}

    \includegraphics[scale = 1]{transducteur_exemple}
    \centering
    
\end{figure}

\end{frame}

\begin{frame}
\frametitle{Lien entre dominos et transducteurs}

On peut voir un pavage comme un transducteur.

En effet, $\forall t = (w,e,s,n) \in T$, on dit qu'il y a une transition de l'état $w$ vers l'état $e$ qui lit $n$ et écrit $s$.



\end{frame}

\end{document}



